\begin{titlepage}
	\titlehead{
		\begin{flushright}
			\includegraphics[width=0.45\textwidth]{img/Logo_Informatik.pdf}
		\end{flushright}
	}

	\subject{\art \\ in der Angewandten Informatik\\ \large{\mdseries{Nr. \registriernr}}}
	
    \title{\Huge\titel}
    
	% \subtitle{\Large\untertitel\\[5em]}
	
    \author{\textbf{\autor}}
	
    \date{Abgabedatum: \datum}
	
    \publishers{\erstgutachter \\ \zweitgutachter}
	
    \maketitle                                
\end{titlepage}

\pagenumbering{Roman}

\begin{abstract}
\section*{Kurzfassung}
Diese Bachelor Arbeit beschäftigt sich mit der Implementierung eines Computergegners für das Brettspiel ``Chamäleon Schach''. Die Umsetzung erfolgt mithilfe von klassischen Algorithmen nach dem sog. heuristischen Ansatz. Besondere Herausforderung dabei ist, dass das Spiel nicht nur zu zweit, sondern auch zu dritt oder zu viert gespielt werden kann. Für Zwei-Spieler-Spiele gibt es einen klaren Standard-Algorithmus, den MiniMax mit Alpha-Beta Pruning. Ab drei Spielern gibt es verschiedene Algorithmen, die geeignet sein könnten. Deshalb werden in dieser Arbeit verschiedene solcher Algorithmen implementiert und miteinander verglichen. Die Algorithmen sind: der Max\textsuperscript{N} in seiner Standard-Implementierung und mit verschiedenen Pruning-Verfahren, der Hypermax und die paranoide Version des MiniMax mit Alpha-Beta Pruning.
\end{abstract}

\begin{abstract}
\section*{Abstract}
This bachelor thesis is about the implementation of a computer opponent for the board game ``chameleon chess''. Classic Algorithms were used to solve this problem, they follow the so-called heuristic approach. The most challenging part is that the game can be played by two, three or four players. For two player games there is a standard algorithm – the MiniMax with Alpha Beta pruning. However, for three players or more there are many different algorithms that could be used. Some of them were implemented and their performance was compared. The algorithms were: Max\textsuperscript{N} in its standard implementation and using different pruning methods, the hypermax and the paranoid version of MiniMax with Alpha Beta pruning.
\end{abstract}